\chapter*{Abstrakt}
\label{sec:Abstrakt}

Das E-Mail-Management ist Teil unseres t"aglichen Lebens. Es werden mehrere Clients genutzt, um regelm"a"sig auf die E-Mails zuzugreifen. Bei der Masse an E-Mails, die wir t"aglich Empfangen, ist es schwierig, den "Uberblick zu behalten. Eine Snooze-Funktionalit"at kann dabei helfen, das Management zu verbessern.

Das Ziel dieser Arbeit ist zu beantworten, welche M"oglichkeiten es gibt, eine Snooze-Funktionalit"at "uber das IMAP-Protokoll interoperabel zu implementieren.

Dazu wurden existierende L"osungen in E-Mail-Clients betrachtet sowie andere M"oglichkeiten untersucht, die das IMAP-Protokoll bietet. Zus"atzlich wurden diese auf die Praxistauglichkeit anhand der Nextcloud Mail-App untersucht.

Die Analyse hat gezeigt, dass es nicht die eine perfekte L"osung f"ur eine interoperable Implementierung gibt. Es wurden mehrere Teill"osungen beschrieben, die in Kombination eine ganzheitliche L"osung bieten k"onnen. Die Nachteile der einzelnen Teill"osungen wurden gegeneinander abgewogen. Als Vorschlag f"ur eine Standard-Erweiterung des IMAP-Protokolls wurde eine Kombination aus der Verwendung von Keywords, dem Nutzen von einer neuen Special-Use-Mailbox und der Verwendung der Quick-Resynchronization-Erweiterung empfohlen.  

Basierend auf diesem Vorschlag kann ein Internet-Draft entstehen und der Standadisierungsprozess f"ur eine Erweiterung beginnen.
