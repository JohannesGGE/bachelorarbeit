\chapter{AT: What is it, How it works}
\label{sec:HowItWorks}

\section{AT: Effizientes Mail Management}
\label{sec:InboxZero}

Viele stehen vor dem Problem, dass ihr E-Mail Posteingang \hl{(im folgenden INBOX genannt)} �berf�llt ist mit E-Mails, welche noch beantwortet werden m�ssen. Das E-Mail Postfach �ffnen f�hrt zu Stress, und es werden immer mehr E-Mails obwohl man sein bestes gibt diese zu bearbeiten. Es scheint so, als ob die F�higkeit mit der Masse an E-Mails klar zu kommen, entscheided ist �ber den Erfolg, den man vorallem im Job hat. Die Inbox wird als To-Do Liste, Kalender und Postfach in einem genutzt. \cite{googletechtalksInboxZero2007} Dieses Konzept f�hrt gezwungener Ma�en zu Problemen. Es gibt aber Konzepte, mit denen dieses Problem angegangen werden k�nnen. Das Ziel ist immer das effiziente abarbeiten von E-Mails.
Eine Methode ist die Inbox Zero, welche von Merlin Mann erstmalig definiert wurde und nicht das Ziel hat keine E-Mails mehr in der Inbox zu haben, wie man es durch den Namen annehmen k�nnte, sondern eher darauf bezogen ist, kein Stress bei der Bearbeitung der Inbox zu haben. \cite{WhatInboxZero2019} Es geht nicht darum alle E-Mails in der Inbox zu beantworten sondern alle E-Mails zu bearbeiten. Das hei�t, dass man sich Aktionen definiert, welche man auf die zu bearbeitende E-Mail anwendet. Das kann das Beantworten der E-Mail sein aber genauso das L�schen oder Zur�ckstellen der E-Mail. Wichtig ist, dass man ohne viel �berlegen entscheidet, welche Aktion auszuf�hren ist und sich nach ausf�hren der n�chsten E-Mail widmet. \cite{googletechtalksInboxZero2007} 

\begin{itemize}
	\item Stress bei Inbox �ffnen, da viel zu viele Mails
	\item Verschiedene Methoden das zu �ndern, welche alle �hnlich sind und das gleiche Ziel verfolgen
	\begin{itemize}
		\item N-days later
		\item Inbox Zero (productivity philosophy about dealing with a constant, life-long stream of email erstmalig von Merlin Mann erdacht) \cite{WhatInboxZero2019}
		\begin{itemize}
			\item Inbox ist nur ein �bergang, aber nicht das Ende bzw. der Endpunkt f�r eine E-Mail \cite{googletechtalksInboxZero2007}
		\end{itemize}
		\item Getting Things Done
		\item To-Do Inbox
	\end{itemize}
\end{itemize}

\section{AT: E-Mail and IMAP}
\label{sec:EmailAndIMAP}


\chapter{AT: Problemstellung in Mail}
\label{sec:Problemstellung}

\begin{itemize}
	\item Wie kann kann Inbox Zero in Mail Clients aussehen?
	\item Warum ist es n�tig das �ber IMAP zu implementieren?
\end{itemize}


\chapter{AT: Exsitierende Implementierungen}
\label{sec:ExistierendeImpl}

\begin{itemize}
	\item Keine Implementierung
	\item Locale Implementierung (Offline)
	\item Environment Implementierung (Apple, Exchange, Gmail...)
	\item Scripts zur Automatisierung
	\item Nur Labeln (Thunderbird)
\end{itemize}


\chapter{AT: Diskussion L�sungsans�tze}
\label{sec:FuenfteUeberschrift}

\section{AT: Allgemein}
\label{sec:LoesungAllgemein}

\begin{itemize}
	\item Keywords
	\item Dublizieren
	\item Metadata Capability
	\item JMAP (nicht IMAP aber, sollte erw�hnt werden)
	\item ...
\end{itemize}

\section{AT: Nextcloud Mail App}
\label{sec:LoesungNextcloud}

\begin{itemize}
	\item Welcher L�sungsansatz passt am besten in die App
\end{itemize}


\chapter{AT: Vorschlag Standarderweiterung}
\label{sec:VorschlagStandard}

\begin{itemize}
	\item Wie k�nnte ein RFC f�r eine Snooze aussehen
\end{itemize}
