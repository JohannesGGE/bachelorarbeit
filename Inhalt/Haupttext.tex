\chapter{AT: What is it, How it works}
\label{sec:HowItWorks}
Text

\section{AT: Konzept Inbox Zero}
\label{sec:InboxZero}
Und hier wird es schon spezifischer.

\section{AT: E-Mail and IMAP}
\label{sec:EmailAndIMAP}


\chapter{AT: Problemstellung}
\label{sec:Problemstellung}

\begin{itemize}
	\item Wie kann kann Inbox Zero in Mail Clients aussehen?
	\item Warum ist es n�tig das �ber IMAP zu implementieren?
\end{itemize}


\chapter{AT: Exsitierende Implementierungen}
\label{sec:ExistierendeImpl}

\begin{itemize}
	\item Keine Implementierung
	\item Locale Implementierung (Offline)
	\item Environment Implementierung (Apple, Exchange, Gmail...)
	\item Scripts zur Automatisierung
	\item Nur Labeln (Thunderbird)
\end{itemize}


\chapter{AT: Diskussion L�sungsans�tze}
\label{sec:FuenfteUeberschrift}

\section{AT: Allgemein}
\label{sec:LoesungAllgemein}

\begin{itemize}
	\item Keywords
	\item Dublizieren
	\item Metadata Capability
	\item JMAP (nicht IMAP aber, sollte erw�hnt werden)
	\item ...
\end{itemize}

\section{AT: Nextcloud Mail App}
\label{sec:LoesungNextcloud}

\begin{itemize}
	\item Welcher L�sungsansatz passt am besten in die App
\end{itemize}


\chapter{AT: Vorschlag Standarderweiterung}
\label{sec:VorschlagStandard}

\begin{itemize}
	\item Wie k�nnte ein RFC f�r eine Snooze aussehen
\end{itemize}


In der Regel reichen zwei Gliederungsebenen f�r Graduierungsarbeiten aus.