\chapter{AT: What is it, How it works}
\label{sec:HowItWorks}

\section{AT: Effizientes Mail Management}
\label{sec:InboxZero}

Viele stehen vor dem Problem, dass ihr E-Mail Posteingang \hl{(im folgenden INBOX genannt)} �berf�llt ist mit E-Mails, welche noch beantwortet werden m�ssen. Das E-Mail Postfach �ffnen f�hrt zu Stress, und es werden immer mehr E-Mails obwohl man sein bestes gibt diese zu bearbeiten. Es scheint so, als ob die F�higkeit mit der Masse an E-Mails klar zu kommen, entscheided ist �ber den Erfolg, den man vorallem im Job hat. Die Inbox wird als To-Do Liste, Kalender und Postfach in einem genutzt. \cite{googletechtalksInboxZero2007} Dieses Konzept f�hrt gezwungener Ma�en zu Problemen. Es gibt aber Konzepte, mit denen dieses Problem angegangen werden kann. Die meisten basieren auf Getting Things Done, was eine Produktivit�tsstrategie ist, die das Ziel hat Aufgaben, Informationen, Probleme oder Projekte zu extern, also nicht im Kopf zu verwalten. \cite{GettingThingsDone2023} David Allen hat diese Methode entwickelt und beschreibt in seinem Buch "Getting Things Done: The Art of Stress-Free Productivity", wie Sie funktioniert und anzuwenden ist. \hl{Buch ref??}
Auf der allgemeinen Getting Things Done Methode basieren andere Strategien, welche sich auf das E-Mail Management spezialisieren. Inbox Zero ist eine davon, welche von Merlin Mann erstmalig definiert wurde und nicht das Ziel hat keine E-Mails mehr in der Inbox zu haben, wie man es durch den Namen annehmen k�nnte, sondern eher darauf bezogen ist, kein Stress bei der Bearbeitung der Inbox zu haben. \cite{WhatInboxZero2019} Es geht nicht darum alle E-Mails in der Inbox zu beantworten sondern alle E-Mails zu bearbeiten. Das hei�t, dass man sich Aktionen definiert, welche man auf die zu bearbeitende E-Mail anwendet. Das kann das Beantworten der E-Mail sein aber genauso das L�schen oder Zur�ckstellen der E-Mail. Wichtig ist, dass man ohne viel �berlegen entscheidet, welche Aktion auszuf�hren ist und sich nach ausf�hren der n�chsten E-Mail widmet. \cite{googletechtalksInboxZero2007}
\hl{�hnlich ist die N days later Methode. (Nicht relevant???)}

\begin{itemize}
	\item Stress bei Inbox �ffnen, da viel zu viele Mails
	\item Verschiedene Methoden das zu �ndern, welche alle �hnlich sind und das gleiche Ziel verfolgen
	\begin{itemize}
		\item Getting Things Done
		\item Inbox Zero (productivity philosophy about dealing with a constant, life-long stream of email erstmalig von Merlin Mann erdacht) \cite{WhatInboxZero2019}
		\begin{itemize}
			\item Inbox ist nur ein �bergang, aber nicht das Ende bzw. der Endpunkt f�r eine E-Mail \cite{googletechtalksInboxZero2007}
		\end{itemize}
		\item To-Do Inbox
		\item (N-days later)
	\end{itemize}
\end{itemize}

\section{AT: E-Mail and IMAP}
\label{sec:EmailAndIMAP}

\begin{itemize}
	\item Wie wird Snooze definiert?
	\item Was genau hei�t interoperabel
\end{itemize}


\chapter{AT: Problemstellung in Mail}
\label{sec:Problemstellung}

\begin{itemize}
	\item Wie kann kann Inbox Zero/GTD in Mail Clients aussehen, bzw. welche M�glichkeiten braucht man um diese Konzepte umsetzen zu k�nnen?
	\item Warum ist es n�tig/sinnvoll das �ber IMAP zu implementieren?
	\begin{itemize}
		\item Interoperabel
	\end{itemize}
\end{itemize}


\chapter{AT: Exsitierende Implementierungen}
\label{sec:ExistierendeImpl}

Bei der Analyse von bestehenden Implementierungen einer Snooze Funktion f�r E-Mails gibt es verschiedene Ans�tze in den E-Mail Clients.
Es gibt viele Clients, die diese Funktion gar nicht bieten. Darunter fallen z.B. der Android E-Mail Client K-9 \hl{LINKs?}, OpenExchangeMail, Snappymail, (Zimbra), Rainloop, Tutanota, Posteo, ProtonMail, Cypht Client oder Evolution.
Bei Clients mit der M�glichkeit zu Snoozen, gibt es verschiedene Ans�tze. Manche implementieren es rein lokal


\begin{itemize}
	\item Keine Implementierung
	\item Locale Implementierung (Offline)
	\begin{itemize}
		\item Mailbird
		\begin{itemize}
			\item TBD
		\end{itemize}
		\item FairEmail
		\begin{itemize}
			\item Android E-Mail Client
			\item Hat snooze Funktion
			\item 'snoozing is done locally on the device by hiding the message while it is snoozing. Unfortunately, it is not possible to hide messages on the email server too. Deleting messages from the server and restoring them later could result in losing messages when the device breaks.' \cite{FairEmailFAQMd}
			\item Bei wiederanzeigen als ungelesen, mit snoozed icon aber keine sortierung an den Anfang
		\end{itemize}
	\end{itemize}
	\item Environment Implementierung (Apple, Exchange, Gmail...)
	\begin{itemize}
		\item Gmail
		\begin{itemize}
			\item Gmail Label und Ordner erkl�ren
			\item Gmail kann seit Mitte 2018 Mails snoozen. \cite{NewGmailNow}
			\item Dabei werden die Mails intern als snoozed makiert und aus der Inbox entfernt (Label entfernt) (wenn aus einer anderen Mailbox (Label Scop) heraus gesnoozed wird, dann bleiben Sie in der Mailbox (Label Scop)). Tauchen dann unter dem Reiter "Zur�ckgestellt" auf mit einem Datum, wann Sie entsnoozed werden. 
			\item Beim entsnoozen wird das Label "Posteingang" gesetzt, nicht ungelesen, aber am Anfang angezeigt
			\item Die gesnoozeden Mails werden �ber IMAP nicht angezeigt (gibt kein �ber IMAP ver�ffentlichten Ordner/Label). Die Mail ist nur noch in dem Ordner "alle emails/Alle nachrichten" enthalten. (Bewerten)
		\end{itemize}
		\item Exchange/Qutlook
		\begin{itemize}
			\item 
		\end{itemize}
		\item Apple-Mail
		\item Mailspring
		\item Fastmail
		\item eM Client
		\item Sark Mail
		\item ZohoMail
		\begin{itemize}
			\item 
		\end{itemize}
	\end{itemize}
	\item Scripts zur Automatisierung
	\begin{itemize}
		\item Python IMAP Snooze
	\end{itemize}
	\item Nur Labeln (Thunderbird)
	\begin{itemize}
		\item Thunderbird
	\end{itemize}
	\item Working Solutions
	\begin{itemize}
		\item Roundcube Plugin
	\end{itemize}
\end{itemize}

\begin{itemize}
	\item Kein IMAP
	\item Warum wurden welche Clients untersucht??
	\item Welche Annahmen wurden �ber propr�rit�re Software getroffen??
\end{itemize}


\chapter{AT: Diskussion L�sungsans�tze}
\label{sec:FuenfteUeberschrift}

\section{AT: Allgemein}
\label{sec:LoesungAllgemein}

\begin{itemize}
	\item Keywords
	\item Dublizieren
	\item Metadata Capability
	\item JMAP (nicht IMAP aber, sollte erw�hnt werden)
	\item ...
	\item Was muss beachtet werden (THREADs, Mail am Anfang stellen) (Vlt Unter�berschriften, der einzelnen Probleme (Daten Speichern, verschieben, An Anfang der INBOX))
\end{itemize}

\section{AT: Nextcloud Mail App}
\label{sec:LoesungNextcloud}

\begin{itemize}
	\item Welcher L�sungsansatz passt am besten in die App
	\item Was kann Dovecot als Server im Hindergrund??
\end{itemize}


\chapter{AT: Vorschlag Standarderweiterung}
\label{sec:VorschlagStandard}

\begin{itemize}
	\item Wie k�nnte ein RFC f�r eine Snooze aussehen
	\item Keywords/Permaflags und Specual Use Mailbox
\end{itemize}
