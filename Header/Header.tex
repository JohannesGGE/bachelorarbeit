%allgemeine Formatangaben
\documentclass[
 a4paper, 					% Papierformat
 11pt,						% Schriftgr��e
 ngerman, 					% f�r Umlaute, Silbentrennung etc.
 titlepage,					% es wird eine Titelseite verwendet
 bibliography=totoc,		% Literaturverzeichnis im Inhaltsverzeichnis auff�hren
 listof=totoc,				% Verzeichnisse im Inhaltsverzeichnis auff�hren
 oneside, 					% einseitiges Dokument
 captions=nooneline,		% einzeilige Gleitobjekttitel ohne Sonderbehandlung wie mehrzeilige Gleitobjekttitel behandeln
 numbers=noenddot,			% �berschriften-??Nummerierung ohne Punkt am Ende
 parskip=half				% zwischen Abs�tzen wird eine halbe Zeile eingef�gt
 ]{scrreprt}

\newcommand{\titel}{Beispieldokument f�r die Erstellung einer Bachelor- oder Masterarbeit mit LaTeX}
\newcommand{\abschlussart}{Master of Science (M.Sc.)}
\newcommand{\arbeit}{Masterarbeit}
\newcommand{\hochschule}{Hochschule f�r Technik, Wirtschaft und Kultur Leipzig}
\newcommand{\fakultaet}{Fakult�t Informatik und Medien}
\newcommand{\autor}{Max Muster}
\newcommand{\studiengang}{Masterstudiengang Medieninformatik}
\newcommand{\matrikelnr}{12345}
\newcommand{\erstgutachter}{Prof. Dr. Musterfrau}
\newcommand{\zweitgutachter}{Prof. Dr. Mustermann}
\newcommand{\ort}{Leipzig}
 	% einbinden von pers�nlichen Daten

%Highlighting
%\usepackage{color}
\usepackage{soul}


% Anpassung an Landessprache
\usepackage[ngerman]{babel}
 
\usepackage[default]{sourcesanspro}
 
% Verwenden von Sonderzeichen und Silbentrennung
\usepackage[latin1]{inputenc}	
\usepackage[T1]{fontenc}			
\usepackage{textcomp} 																% Euro-Zeichen und andere
\usepackage[babel,german=quotes]{csquotes}						% Anf�hrungszeichen
\RequirePackage[ngerman=ngerman-x-latest]{hyphsubst} 	% erweiterte Silbentrennung

% Befehle aus AMSTeX f�r mathematische Symbole z.B. \boldsymbol \mathbb
\usepackage{amsmath,amsfonts}

% Zeilenabst�nde und Seitenr�nder 
\usepackage{setspace}
\usepackage{geometry}

% Einbinden von JPG-Grafiken
\usepackage{graphicx}

% zum Umflie�en von Bildern
% Verwendung unter http://de.wikibooks.org/wiki/LaTeX-Kompendium:_Baukastensystem#textumflossene_Bilder
\usepackage{floatflt}

% Verwendung von vordefinierten Farbnamen zur Colorierung
% Palette und Verwendung unter http://kitt.cl.uzh.ch/kitt/CLinZ.CH/src/Kurse/archiv/LaTeX-Kurs-Farben.pdf
\usepackage[usenames,dvipsnames]{color} 

% Tabellen
\usepackage{array}
\usepackage{longtable}

% einfache Grafiken im Code
% Einf�hrung unter http://www.math.uni-rostock.de/~dittmer/bsp/pstricks-bsp.pdf
\usepackage{pstricks}

% Quellcodeansichten
\usepackage{verbatim}
\usepackage{moreverb} 											% f�r erweiterte Optionen der verbatim Umgebung
% Befehle und Beispiele unter http://www.ctex.org/documents/packages/verbatim/moreverb.pdf
\usepackage{listings} 											% f�r angepasste Quellcodeansichten siehe
% Kurzeinf�hrung unter http://blog.robert-kummer.de/2006/04/latex-quellcode-listing.html

% Glossar und Abbildungsverzeichnis
\usepackage[
nonumberlist, %keine Seitenzahlen anzeigen
acronym,      %ein Abk�rzungsverzeichnis erstellen
toc          %Eintr�ge im Inhaltsverzeichnis
]      %im Inhaltsverzeichnis auf section-Ebene erscheinen
{glossaries}

% verlinktes und Farblich angepasstes Inhaltsverzeichnis
\usepackage[pdftex,
colorlinks=true,
linkcolor=InterneLinkfarbe,
urlcolor=ExterneLinkfarbe,
citecolor=Citefarbe]{hyperref}
\usepackage[all]{hypcap}

% URL verlinken, lange URLs umbrechen
\usepackage{url}

% sorgt daf�r, dass Leerzeichen hinter parameterlosen Makros nicht als Makroendezeichen interpretiert werden
\usepackage{xspace}

% Beschriftungen f�r Abbildungen und Tabellen
\usepackage{caption}

% Entwicklerwarnmeldungen entfernen
\usepackage{scrhack}

% Keine Worttrennungen
\usepackage[none]{hyphenat}
\sloppy

% PDF/A erzeugen
%\usepackage[a-1b]{pdfx}		% einbinden der verwendeten Latex-Pakete

\singlespacing				% einfacher Zeilenabstand
%\onehalfspacing 			% 1,5facher Zeilenabstand

\definecolor{InterneLinkfarbe}{rgb}{0.1,0.1,0.3} 	% Farbliche Absetzung von externen Links
\definecolor{ExterneLinkfarbe}{rgb}{0.1,0.1,0.7}	% Farbliche Absetzung von internen Links
\definecolor{Citefarbe}{rgb}{0.0,0.5,0.5}

% Einstellungen f�r Fu�noten:
\captionsetup{font=footnotesize,labelfont=sc,singlelinecheck=true,margin={5mm,5mm}}

% Stil der Quellenangabe
\bibliographystyle{alphadin}
%\bibliographystyle{ACM-Reference-Format}

%Ausschluss von Schusterjungen
\clubpenalty = 10000
%Ausschluss von Hurenkindern
\widowpenalty = 10000

% Befehle, die Umlaute ausgeben, f�hren zu Fehlern, wenn sie hyperref als Optionen �bergeben werden
\hypersetup{
%    pdftitle={\titel \untertitel},
%    pdfauthor={\autor},
%    pdfcreator={\autor},
%    pdfsubject={\titel \untertitel},
%    pdfkeywords={\titel \untertitel},
}

% Beispiel f�r eine Listings-Codeumbebungen
% Bei mehreren Definitionen empfielt sich das auslagern in eine externe Datei
\lstloadlanguages{Java,HTML}
\lstset{
	frame=tb,
	framesep=5pt,
	basicstyle=\footnotesize\ttfamily,
	showstringspaces=false,
	keywordstyle=\ttfamily\bfseries\color{CadetBlue},
	identifierstyle=\ttfamily,
	stringstyle=\ttfamily\color{OliveGreen},
	commentstyle=\color{GrayBlue},
	rulecolor=\color{Gray},
	xleftmargin=5pt,
	xrightmargin=5pt,
	aboveskip=\bigskipamount,
	belowskip=\bigskipamount
} 

%Den Punkt am Ende jeder Beschreibung deaktivieren
\renewcommand*{\glspostdescription}{}

% Empfehlung: Abkuerzungsverzeichnis und Glossar sind in Graduierunsarbeiten
% nicht zwingend notwendig

% %Glossar-Befehle anschalten
% \makeglossaries
% \glsenablehyper
% \input{Header/Abkuerzungen}
% \input{Header/Glossar}


% --------- Seitenlayout -----------

% Zwischenabgabe
\geometry{
	top=2cm,
	left=3cm,
	right=3cm
}

% Endabgabe
%\areaset[0.5cm]{16cm}{24cm}